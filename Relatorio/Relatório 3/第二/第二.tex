\pagebreak
\section{Etapa Atual}

No momento, com o auxílio do professor orientador estudei conteúdos relacionado ao uso de dados de acelerômetro
em reconhecimento de padrões.
Dentre esses materiais temos a base de dados de um acelerômetro\cite{AccelerometerData} no repositório da UCI para predizer sobre vibração de máquinas elétricas, o qual estou tentando implementar no espGG. 

% Na tentativa de implementar essa base de dados no algoritmo espGG me deparei com alguns problemas. Primeiramente, o
% código em C++ utiliza a variável ArrayXXd da biblioteca Eigen para os cálculos internos. Porém esse tipo de variável suporta
% apenas dados de entrada de dimensão 2. Os dados de acelerômetro disponibilizados nessa base apresenta dimensão 3, sendo eles a aceleração
% nos eixos x,y e z.

Um ponto de atenção é com relação à memória e processamento limitados do microcontrolador. Por esse motivo uma etapa importante do trabalho
será a de realizar o pré-processamento dos dados de forma a otimizar o algoritmo e reduzir a necessidade desses recursos.


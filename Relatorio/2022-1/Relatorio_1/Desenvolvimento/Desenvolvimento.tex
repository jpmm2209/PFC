\section{Desenvolvimento}

\subsection*{3.1 Análise visual}

Foram recolhidos dados de amostra para:

- vibração nos eixos X, Y e Z

- movimento circular no eixo Z

- movimento de rotação em torno de um ponto fora do objeto no eixo Y. (Gira e volta)

A aquisição de dados foi feita num período de 50ms.

Nas figuras abaixo vemos os dados recolhidos. Em cada imagem temos 6 gráficos referentes respectivamente à aceleração 3 eixos e giroscópio nos 3 eixos.
Podemos também ver que tivemos etapas de resolução do movimento e etapas de repouso.


\begin{figure}[H]
    \center
    \includegraphics[width=7cm]{images/VibracaoX.png}
    \label{img6}
    \caption{Vibração no eixo X}
\end{figure}

\begin{figure}[H]
    \center
    \includegraphics[width=7cm]{images/VibracaoY.png}
    \label{img2}
    \caption{Vibração no eixo Y}
\end{figure}

\begin{figure}[H]
    \center
    \includegraphics[width=7cm]{images/VibracaoZ.png}
    \label{img3}
    \caption{Vibração no eixo Z}
\end{figure}

\begin{figure}[H]
    \center
    \includegraphics[width=9cm]{images/CirculoZ.png}
    \label{img4}
    \caption{Movimento circular no eixo Z}
\end{figure}

\begin{figure}[H]
    \center
    \includegraphics[width=9cm]{images/GiraeVoltaY.png}
    \label{img5}
    \caption{ Gira e Volta no eixo Y }
\end{figure}




\subsection*{3.2 Tratamento dos dados recolhidos}

Primeiramente foi classificado cada comportamento de curva de acordo com seu 
respectivo movimento associado.

Com isso resultamos com um arquivo csv onde cada linha é um valor amostrado no tempo e 
contendo as seguintes colunas:

- AccX (aceleração no eixo X)

- AccY (aceleração no eixo Y)

- AccZ (aceleração no eixo Z)

- AngX (rotação no eixo X)

- AngY (rotação no eixo Y)

- AngZ (rotação no eixo Z)

- Class (classificação do movimento)

Abaixo vemos uma amostra de cada tipo de movimento classificado.

\begin{figure}[H]
    \center
    \includegraphics[width=6cm]{images/sampleVibracaoX.png}
    \caption{ sample VibracaoX }
\end{figure}
\begin{figure}[H]
    \center
    \includegraphics[width=6cm]{images/sampleVibracaoY.png}
    \caption{ sample VibracaoY}
\end{figure}
\begin{figure}[H]
    \center
    \includegraphics[width=6cm]{images/sampleVibracaoZ.png}
    \caption{ sample VibracaoZ}
\end{figure}
\begin{figure}[H]
    \center
    \includegraphics[width=6cm]{images/sampleCirculoZ.png}
    \caption{ sample CirculoZ }
\end{figure}
\begin{figure}[H]
    \center
    \includegraphics[width=6cm]{images/sampleGiraeVoltaY.png}
    \caption{ sample GiraeVoltaY }
\end{figure}
\begin{figure}[H]
    \center
    \includegraphics[width=6cm]{images/sampleNoise.png}
    \caption{ sample Noise }
\end{figure}
\begin{figure}[H]
    \center
    \includegraphics[width=6cm]{images/sampleParado.png}
    \caption{ sample Parado }
\end{figure}

Analisando visualmente já podemos perceber que:

- sinais com maior amplitude alta e frequência alta tendem a ser movimentos de vibração.
Definindo qual eixo temos uma amplitude maior que as demais sabemos qual eixo o movimento foi destinado.

- sinais com amplitude alta em baixas frequências tendem a ser movimento de envolvem rotação.
Definindo o formato da onda conseguimos diferenciar movimento de GiraeVolta de movimentos de Circulo.

- Sinais estáveis definem se o objeto está parado. De acordo com seus valores também podemos saber a posição de repouso.

- Sinais ruidosos são bem comuns em sensores MEMS e por isso não podem ser associados à algum tipo de movimento.




\subsection*{3.2 Análise estatística dos dados}

Primeiramente analisei a distribuição de amostras que temos.
Podemos ver no gráfico de Figura 14 que essa distribuição está desbalanceada.
Temos mais amostras de movimento parado.

\begin{figure}[H]
    \center
    \includegraphics[width=9cm]{images/sampleDistribution.png}
    \caption{ sample Distribution }
\end{figure}

Também foi realizado a distribuição de densidade dos dados em cada eixo do sensor 
para todos os movimentos em questão.

As Figuras 15, 16 e 17 representam os dados recolhidos nos eixos do acelerômetro.
Nelas vemos que o movimento de vibração nos eixos correspondentes apresentam dados muito mais distribuidos.


\begin{figure}[H]
    \center
    \includegraphics[width=7cm]{images/AccXdistribution.png}
    \caption{ AccX distribution }
\end{figure}
\begin{figure}[H]
    \center
    \includegraphics[width=7cm]{images/AccYdistribution.png}
    \caption{ AccY distribution }
\end{figure}
\begin{figure}[H]
    \center
    \includegraphics[width=7cm]{images/AccZdistribution.png}
    \caption{ AccZ distribution }
\end{figure}

As Figuras 18, 19 e 20 representam os dados recolhidos nos eixos do giroscópio.
Podemos ver que nos dados desse sensor temos maior separabilidade entre os movimentos de rotação.

\begin{figure}[H]
    \center
    \includegraphics[width=5cm]{images/AngXdistribution.png}
    \caption{ AngX distribution }
\end{figure}
\begin{figure}[H]
    \center
    \includegraphics[width=5cm]{images/AngYdistribution.png}
    \caption{ AngY distribution }
\end{figure}
\begin{figure}[H]
    \center
    \includegraphics[width=5cm]{images/AngYdistribution.png}
    \caption{ AngY distribution }
\end{figure}




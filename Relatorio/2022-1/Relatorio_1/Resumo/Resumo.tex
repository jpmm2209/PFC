\section{Resumo}

Esse trabalho de projeto de final de curso se dedica à implementação do algoritmo de reconhecimento de padrões por
meio de grafo geométrico, utilizando-se especificamente do Grafo de Gabriel.\cite*[]{LuizBambirra}.

A aplicação será classificar movimentos de um objeto. Para isso faz-se o uso de uma placa de desenvolvimento do  microcontrolador ESP32 e um sensor 
acelerômetro a ser escolhido. O protótipo desenvolvido deverá reconhecer qual movimento foi aplicado ao objeto e exercer uma atuação a partir disso.

No semestre anterior validamos o uso do algoritmo do grafo de Gabriel para a separação de regiões em duas dimensões. Para essa validação foi utilizado o 
exemplo de teste utilizado na própria documentação do grafo e obtivemos os mesmos resultados. 

Nesse projeto a quantidade de dimensões a ser utilizada nas amostras de dados será maior. O grafo de Gabriel
apresenta complexidade exponencial com relação a dimensão do dado de entrada. Por esse motivo a ideia é fazer o treinamento do grafo
em um computador pessoal e implementa-lo no microcontrolador para fazer as devidas inferências.


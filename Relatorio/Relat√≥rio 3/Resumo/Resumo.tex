\section{Resumo}

Esse relatório tem como objetivo expor as pesquisas e atividades realizadas nessa terceira etapa do PFC1.

O grafo de Gabriel\cite{GabrielGraph1} é um método de reconhecimento de padrões que separa regiões de dados de forma geométrica. O grafo foi 
desenvolvido pela comunidade da UFMG e demonstrou boa acurácia e baixo uso de recursos computacionais.
Por esse motivo foi escolhido como modelo de reconhecimento de padrões a ser utilizado.

A ideia da aplicação a ser implementada é de construir um protótipo de um dispositivo wireless com bateria que possa 
ser um objeto paupável e faça reconhecimento de padrões de movimento do objeto. Para isso será utilizado um microcontrolador ESP32 
e um sensor acelerômetro MPU6050.

Essa aplicação pode ser realizável com a utilização do grafo de Gabriel se mapearmos geometricamente os dados do sensor acelerometro por meio de uma
série temporal da aquisição. Reconhecendo assim o movimento realizado em um certo espaço de tempo pré-definido.
% \cite{LivroTexto}

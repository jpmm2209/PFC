\chapter{Introdução}
%\markboth{\thechapter ~~~ Introdução}{}
%\label{intro}

Se preferir, você pode apresentar este Capítulo antes da primeira Seção, destacando os principais pontos que são abordados. %\cite{Raffo2008}

\section{Motivação e Justificativa}
\markright{\thesection ~~~ Motivação}
%\label{motiva}


\section{Objetivos do Projeto}
\markright{\thesection ~~~ Objetivos}
%label{objetivos}

Tendo em vista o exposto acima, este projeto tem por objetivos:

\begin{itemize}
\item[a.] Item 1;
\item[b.] Item 2;
\item Etc.
\end{itemize}




\section{Local de Realização}
\markright{\thesection ~~~ A Empresa}
%\label{empresa}

Vale à pena descrever a empresa onde o PFC foi desenvolvido. Veja o exemplo abaixo.

O projeto de fim de curso foi desenvolvido na empresa ..., no Departamento de ..., responsável por toda a implementação do sistema de ...

A empresa realiza projetos de pesquisa e desenvolvimento, consultoria e treinamento nas áreas de ...

A ... foi criada em ...  

A empresa é divida em três departamentos: (o arquivo Introducao.tex mostra como criar a lista abaixo) 

\begin{itemize}
\item Departamento de ... 
\item Departamento de ...
\item Departamento de ...
\end{itemize}

Este projeto foi desenvolvido no Departamento de ..., que é o responsável por ...

Os demais Departamentos englobam as fun\c{c}\~oes de ...

Todos os departamentos trabalham em conjunto. O Departamento de ..., por exemplo, precisa manter um grande vínculo com o Departamento de ... Isso ocorre porque todas as especifica\c{c}\~oes de hardware e sistemas influenciam a forma de implementa\c{c}ão de servi\c{c}os, organiza\c{c}\~ao de tabelas e recursos disponíveis.




\section{Estrutura da Monografia}
\markright{\thesection ~~~ Organização do Trabalho}
%\label{organiza}

O trabalho está dividido em quatro capítulos. Este capítulo apresentou uma introdução ao projeto a ser descrito nesta monografia e a empresa onde o trabalho foi realizado. O Capítulo 2 descreve os princípios básicos de um sistema ... (sistema onde se insere o trabalho) e abrange todos os conceitos necessários para um melhor entendimento do projeto. O Capítulo 3 aborda a metodologia de desenvolvimento, seguida pela implementa\c{c}\~ao dos .... No Capítulo 4 tem-se a conclusão da  monografia e algumas sugest \~oes e dificuldades encontradas na realização do projeto.


\clearpage
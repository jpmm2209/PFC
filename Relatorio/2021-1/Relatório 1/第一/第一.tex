% \section{Introdução a Redes Neurais Artificiais}

Dentre as soluções já existentes que serviram de inspiração temos o projeto Magic Wand desenvolvido pela comunidade do TensorFlow Lite para exemplificar o uso da biblioteca de Machine Learning para microcontroladores.
Há também uma versão do autor andriyadi Andri (github: @andriyadi) \footcite{\url{https://github.com/andriyadi/MagicWand-TFLite-Arduino}} que faz 
uma reformulação do exemplo Magic Wand implementado no framework de desenvolvimento para embarcados que pretendemos utilizar nesse trabalho PlatformIO.
Ambos projetos serviram de inspiração para a aplicação escolhida por permitirem reconhecer padrões de movimento por meio de machine learning, porém o método implementado será outro.
Enquanto esses projetos utilizam modelos de alto nível o objetivo nesse trabalho é de implementar as equações do algoritmo do Grafo de Gabriel utilizando bibliotecas de Álgebra Linear em C++.
Na Plataforma CRAN\cite*{CRAN} há a implementação do grafo desenvolvido pela equipe do laboratório de inteligência computacional da UFMG para sua utilização no programa R\cite*{Rlanguage}. Esse algoritmo servirá como inspiração para
a implementação do mesmo em C++ para embarcados.
Já o artigo do autor Luiz Bambirra\cite*{LuizBambirra} demonstra e exemplifica o uso de Grafo de Gabriel para modelagem dos padrões de entrada nos problemas de classificação.
O modelo é capaz de solucionar problemas de redes neurais que tem por objetivos encontrar uma solução  ótima para maior capacidade de generalização e menor complexidade do algoritmo.
Sendo assim problemas classificados como multiobjetivos. Problemas assim costumam ter não só uma solução ótima, mas um conjuntode soluções ótimas, denominado conjunto Pareto-Ótimas(PO). 
O algoritmo consiste nas etapas de construção geométrica do grafo sobre as amostras, remoção dos ruídos e, em seguida, detecção de bordas. Por fim, define-se a região de separação entre os conjuntos. A proposta desse trabalho é de realizar esse algoritmo no microcontrolador embarcado.

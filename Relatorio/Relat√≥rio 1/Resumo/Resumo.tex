% \section{Resumo}

Com o objetivo de concretizar estudos nas áreas de Redes Neurais Artificiais e
em Arquitetura e Organização de Computadores esse projeto final de curso tem por finalidade
implementar um algoritmo de reconhecimento de padrões para solucionar alguma aplicação prática utilizando um microcontrolador.
O algoritmo selecionado é o Grafo de Gabriel\cite{GabrielGraph1} e o hardware utilizado será um microcontrolador ESP32. O grafo foi 
desenvolvido pela comunidade da UFMG e demonstrou boa acurácia e baixo uso de recursos computacionais para separa regiões de dados por método geométrico.
A ideia da aplicação a ser implementada é de construir um protótipo de um dispositivo wireless com bateria que possa 
ser um objeto paupável e faça reconhecimento de padrões de movimento do objeto. Para isso será utilizado um microcontrolador ESP32 
e um sensor acelerômetro MPU6050. A aplicação pode ser realizável com a utilização do grafo de Gabriel se mapearmos geometricamente os dados do sensor acelerômetro por meio de uma
série temporal da aquisição. Reconhecendo assim o movimento realizado em um certo espaço de tempo pré-definido.



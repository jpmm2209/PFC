\addcontentsline{toc}{chapter}{Resumo}

\begin{center}
\huge{{\bf Resumo}}
\vspace{2cm}
\end{center}

No Resumo, em uma única página, em no máximo dois parágrafos, você explicita os seguintes itens: objetivos do projeto e descrição sucinta do local onde ele foi desenvolvido; metodologia utilizada; e resultados alcançados. Leitores experientes decidem se prosseguirão para a leitura do texto completo após lerem o resumo, a conclusão e a introdução. Por isso nestes lugares você deve colocar um esforço maior de convencimento. Além disso, a linguagem utilizada deve ser acessível a leitores com pouca familiaridade com a área, limitando-se o uso de jargões.
 
\begin{sloppypar}
Este novo parágrafo serve para mostrar que ao pular uma ou mais linhas no texto do arquivo .tex, o \TeX\ entende que você está iniciando outro parágrafo. O comando \textsf{sloppypar} força o texto a não ultrapassar as margens. Só deve ser usado se este problema ocorrer.
\end{sloppypar}

 
\clearpage
\thispagestyle{empty}
\cleardoublepage


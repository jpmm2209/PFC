\addcontentsline{toc}{chapter}{Resumo}

\begin{center}
\huge{{\bf Resumo}}
\vspace{2cm}
\end{center}

No Resumo, normalmente em uma única página, você escreve um parágrafo para cada um dos seguintes itens: objetivos do projeto e descrição sucinta do local onde ele foi desenvolvido; 
metodologia utilizada; e resultados alcançados.
 
\begin{sloppypar}
Este novo parágrafo serve para mostrar que ao pular uma ou mais linhas no texto do arquivo .tex, o \TeX entende que você está iniciando outro parágrafo. O comando sloppypar força o texto a n\~ao ultrapassar as margens. S\'o deve ser usado se este problema ocorrer. kkkkkkkkkkkkkkkkkkkkkkkkkkkkkkkkkkkkkkkkkkkkkkkkkkkkkkkkkkkkkkkkkkkkkkkkkkkkk 
\end{sloppypar}

 
\clearpage
\thispagestyle{empty}
\cleardoublepage


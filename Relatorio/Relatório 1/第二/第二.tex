% \section{Validação do Hardware}

O artigo dos autores Arias-Garcia\cite*{HardwareGabrielGraph} tem uma abordagem similar ao objetivo desse trabalho. Nele foi analisada a desempenho de um classificador utilizando Grafo de Gabriel implementado em Hardware Embarcado. A aplicação foi implementada em um FPGA\footcite[]{FPGA: Field-programmable gate array.}, onde se  é capaz de reorganizar blocos lógicos de processamento de forma a obter um Hardware dedicado para
aquela tarefa. Como a implementação de Redes Neurais requer elevado cálculo matricial e paralelo, a implementação em Hardware dedicado para tal modelo apresenta bom desempenho quando se comparado à implementação em software de alto nível.
Esse artigo se difere com projeto desejado no que se refere ao Hardware escolhido. No artigo foi-se utilizado um hardware capaz de processar a rede de forma mais performática.
No projeto proposto nesse trabalho o hardware utilizado será de Arquitetura ARM\footcite[]{ARM is a family of reduced instruction set computer (RISC) instruction set architectures for computer programming.} com baixa capacidade de processamento. 
Por esse motivo, o objetivo não é ter um resultado melhor do que os implementados em computadores pessoais mas sim em obter um resultado aceitável para a utilização em uma aplicação de baixo custo.
% \cite{LivroTexto}
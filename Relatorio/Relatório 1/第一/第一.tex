\section{Introdução a Redes Neurais Artificiais}

Estudos ao livro \emph{Redes neurais artificiais: teoria e aplicações} \cite{LivroTexto} 
foram realizados de forma a gerar maior compreensão sobre o assunto. A princípio, foi estudado o primeiro
modelo matemático de neurônios artificiais, o neurônio MCP\footcite{MCP: McCulloch-Pitts Neuron — Mankind's First Mathematical Model Of A Biological Neuron.}.
Em seguida, implementados na linguagem python o modelo de neurônio de aproximação linear Adaline \footcite{Adaline: Adaptive Linear Neuron or later Adaptive Linear Element}
e de um modelo de Adaline com uma função de ativação, tornando um neurônio classificador
com melhores resultados quando comparado ao MCP,o neurônio de classificação Perceptron Simples.

Entender o funcionamento do modelo computacional de um neurônio permite também melhor compreensão dos estudos de 
redes de neurônios. Para a aplicação a ser desenvolvida a ideia inicial é de detecção facial por meio de reconhecimento de imagens. Portanto,
estudos sobre redes neurais convolucionais (CNN)\footcite{CNN: Convolutional Neural Networks} estão sendo feitos por se tratar
de um modelo de rede utilizado para tratamento de imagens.
